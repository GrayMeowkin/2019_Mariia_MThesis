% !TeX root = ../main.tex
% Add the above to each chapter to make compiling the PDF easier in some editors.

\chapter{Conclusion}\label{chapter:conclusion}

\section{Future work}
\subsection{Big experiments prediction}
The thesis contains the description of the small text experiment that is supposed to be different from the actual studies upon the MaCon approach. As the goal of the test experiment was to only emulate the actual study and see how the participants react on the integrated data collection, the experiment was much shorter and simpler than the actual studies and this is the main difference.\\  

We suppose that the big experiments will not have a much bigger quantity of the participants, but will have a more complicated case, which would include system development from scratch, because it would allow to collect data throughout all the engineering process. Therefore the experiment would have a more complex system of phases and activities, which will result in using more timepoint and time interval bookmarks to mark certain activities/states of the system, otherwise the big experiment wouldn't differ from the test experiment described in the thesis. Thus, the data collection approach and the analysis tool will be suitable for the bigger experiment as well as it is suitable for the test experiment. \\

\subsection{Possible improvements}
In the terms of data collection there can be minor improvements done to increase the usability of bookmarks ans surveys and media files collection. \\

In one of the recordings the web camera was adjusted too high for the participant and her face wasn't entirely in the frame most of the time. Having a small window to show the image from the webcam will keep the participant aware of the recording and how the image looks there, so the participant would adjust the camera or change the position to be in the frame. Two of the participants stated they would like to turn the recording on manually and three stated, that they would like it to start automatically and be able to turn it off. This is a controversial question, because if the participants can turn the recording off, some important data might be lost. The perfect solution would be to add a functionality to turn it off/on and a setting, where the researcher could enable or disable this option.\\

According to the feedback of one of the participants the time interval states weren't noticeable enough, so adding icons would improve it. As  the experiment-related functionality is integrated into the workbench, it might be useful to integrate instructions for the experiment task, so the researcher wouldn't have to explain them. The instructions might contain interactive guide through the interface and short tutorial. It would be also useful to provide functionality for the researcher to conduct the experiment through the workbench, meaning having two roles: researcher and participant. The functionality can start with integration of the means for to submit the observations and end with remote control over some elements, like marking task as done, highlighting bookmarks, triggering certain event with a survey.\\

As for Analysis tool, there is some functionality that might be useful for the analysis of the experiment data. Adding custom bookmarks would improve the flow of the retrospective analysis as the researcher would be able to mark certain situation he finds important. Ability to select the sessions for the \textit{Overview} tab would improve the performance of the player and make the navigation over the timeline easier because the researcher can select a shorter time range for display.\\ 

Report generation would be interesting for extracting either the plain data either the visualizations from the \textit{Statistics}. Data correlation analysis would be interesting for correlation studies. As for \textit{Text Answers} textual analysis and key words extraction would be interesting, although it's debatable whether it would be useful for the researcher.\\  

\section{Summary}
The goal of this thesis was to develop a methodology to evaluate tool-based engineering approaches on the example of the MaCon approach. The thesis offers to evaluate the approach by designing a user study according to the evaluation goal and hypothesis, to conduct the user study, collecting the experiment data via the integrated data collection means and to analyze the data in the Analysis Tool.\\

The thesis presents the advantages of the integrated data collection comparing to standalone usability data logging solutions and using separate recording and survey tools and provides the methodology and means to integrate the data collection into the MaCon workbench. The data collected during the experiment includes media files, bookmarks, surveys.\\

In order to perform the evaluation of the MaCon approach the thesis provides a solution called Analysis Tool, that provides the functionality to perform the evaluation of the engineering approach in the terms of engineering process, modeling technique and prototyping tool.\\

The tool smoothly integrates the data collected during the experiment and allows to perform retrospective analysis of the experiment, review the statistical results of the surveys and categorize the free form feedback, which is used to make assessment of the MaCon approach and its elements.\\

The thesis describes the test study, that uses the method. The experiment showed that the integrated data collection is useful and comfortable in the terms of media files collection and saving questionnaires results and events. The Analysis Tool is a convenient solution that provides the functionality needed for the engineering approach evaluation. Since there was experiment data imported to the Analysis Tool, the thesis showed on the example how the researcher would use the tool for the engineering approach assessment.\\ 

In a nutshell the thesis shows that an integrated area-specific solution for experimental evaluation of a tool based engineering approach provides a good instrument for process specific information collection. A suitable analysis program designed specifically to work with the format of the data and tailored distinctively for the types of the tasks the researcher faces during the analysis of the process, model and tool can be a solution, that includes all the needed functionality without redundant complexity.  


