% !TeX root = ../main.tex
% Add the above to each chapter to make compiling the PDF easier in some editors.


\chapter{Curvature computation}\label{chapter:curvature_computation}

In this chapter, we present the reader to the knowledge needed for a better understanding of the problem scope and solution approach and introduce the concept of an \textit{Angular Curvature Profile}. Since we offer to estimate the curvature based investigation to find Landmark points as well as characteristics points on the triangular surface, this chapter gives a description of required techniques and methods. 

\section{Principal curvature calculation }\label{section:princip_curvature}

Consider to determinate the principal curvature of the surface at a vertex \textbf{\textit{s}}. For this purpose assume the firs-order cluster consisting of the adjacent triangles. In a similar manner constuct the  



\section{Evaluation of the engineering process}\label{section:eval_e_p}



\section{Evaluation of the modelling technique}\label{section:eval_m_t}

%----------------------------------------------------------------
\section{Evaluation of the engineering tool}\label{section:eval_p_t}
%----------------------------------------------------------------
\section{Evaluation methodology}\label{section:eval_methodology}
%----------------------------------------------------------------
\subsection{User research}\label{subsection:user_study}
%----------------------------------------------------------------
\subsection{Retrospective analysis}\label{subsection:retrospective}
